\documentclass[12pt,a4paper,oneside,onecolumn,titlepage]{book}
\usepackage[utf8]{inputenc}
\usepackage[spanish]{babel}
\usepackage{amsmath}
\usepackage{amsfonts}
\usepackage{amssymb}
\author{Ángel González}
\title{Apuntes de Estadística 2017}
\begin{document}
\chapter{Probabilidad}
\section{Definiciones preliminares}
\paragraph{Experimento aleatorio:}
Es aquel que, bajo las mismas condiciones iniciales, puede presentar varios resultados diferentes sin que pueda predecirse en cada caso el resultado que se obtendrá.
\subparagraph{}
Para que un experimento sea aleatorio debe verificar:
\begin{itemize}
\item Que pueda ser repetido, en las mismas condiciones, un número ilimitado de veces.
\item Se conozca de antemano los posibles resultados del experimento.
\item En cada repetición no es posible conocer el resultado que se obtendrá.
\end{itemize}
\paragraph{Espacio muestral asociado a un experimento aleatorio:}
Es el conjunto formado por todos los posibles resultados del experimento aleatorio. Lo designamos con la letra $E$.
\paragraph{Suceso:} Se denomina así a cada uno de los subconjuntos del espacio muestral.
\subparagraph{}
Existen dos tipos de sucesos según su composición:
\begin{itemize}
\item \textbf{Suceso elemental:} Cada uno de los resultados del experimento.
\item \textbf{Suceso compuesto:} El formado por la unión de sucesos elementales.
\end{itemize}
\paragraph{Suceso seguro:}
Es aquel que se verifica siempre. Designado por $E$.
\paragraph{Suceso imposible:}
Es aquel que no se verifica nunca. Lo designamos por $\Phi$.
\paragraph{Espacio de sucesos:}
Es el conjunto formado por todos los sucesos compuestos, incluido el suceso imposible. Se designa por $\beta$.


\section{Operaciones con sucesos}
Sean $A,B \in \beta$:
\paragraph{Suceso unión:} Es el suceso que resulta cuando ocurre $A$, ocurre $B$ o ambos a la vez. Lo designamos por $A \cup B$.
\paragraph{Suceso intersección:} Es el suceso que resulta cuando ocurren $A$ y $B$ a la vez. Lo designamos con $A \cap B$.
\paragraph{Inclusión de sucesos:} Diremos que $A \subset B$ si siempre que ocurre $A$ ocurre $B$. Si $A \subset B$, entonces $A \cap B = A$ y $A \cup B = B$.
\paragraph{Igualdad de sucesos:}Diremos que $A = B$ si $A \subset B$ y $B \subset A$.
\paragraph{Sucesos incompatibles:} Dos sucesos son incompatibles cuando al verificarse uno, no puede hacerlo el otro, su intersección es $\Phi$.
\paragraph{Suceso contrario:} El suceso contrario de $A$ es el que se verifica cuando no lo hace $A$. Se verifica que $A \cap \overline{A} = \Phi$ y $A \cup \overline{A} = E$.
\paragraph{Suceso diferencia:} Se designa por $B - A$. Es el formado por todos los sucesos elementales que pertenecen a $B$ pero no a $A$. Se verifica que $B - A = B \cap \overline{A}$. Además $\overline{A} = E - A$.


\section{Leyes de Morgan}
Sean $A, B \in \beta$:
\begin{itemize}
\item $\overline{A \cup B} = \overline{A} \cap \overline{B}$
\item $\overline{A \cap B} = \overline{A} \cup \overline{B}$
\end{itemize}


\section{Álgebra de sucesos}
Todas las operaciones anteriores se pueden realizar porque $\beta$ tiene estructura de álgebra (es un álgebra de Boole) cuando $E$ es finito o de $\sigma$-álgebra cuando $E$ es infinito.


\section{Definición axiomática de probabilidad}
Diremos que una aplicación $P: \beta \rightarrow \mathbb{R}$ es una probabilidad si verifica:


\end{document}
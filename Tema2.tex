\chapter{Variables aleatorias.}
\section{Definición de variable aleatoria.}
\paragraph{}
Una variable aleatoria es una aplicación $X:E\rightarrow\mathbb{R}$ que a cada elemento del espacio muestral le hace corresponder un número real.

\paragraph{}
La definición solo tiene sentido si:
\begin{itemize}
\item Para cada $\omega \in E$ se verifica que $X(\omega)\in\mathbb{R}$.
\item Si $B=(-\infty,x]\in\mathbb{R}, X^{-1}(B)\in\beta$.
\end{itemize}

También una variable aleatoria es una cantidad variable cuyos valores dependen del azar y para la cual existe una distribución de probabilidad.


\section{Función de distribución de una v.a.}
Sea $(E,\beta,P)$ un espacio probabilístico asociado a un experimento aleatorio y sea $X:E\rightarrow\mathbb{R}$ una variable aleatoria. Se llama función de distribución de $X$ a la aplicación $F:\mathbb{R}\rightarrow\mathbb{R}$ que a cada $x\in\mathbb{R}$ le hace corresponder:
\[F(x)=P(X \leq x)=P(X\in(-\infty,x])\]

\subsection{Propiedades.}
\begin{itemize}
\item[1.]$\lim_{x \to -\infty} F(x)=0$; $\lim_{x \to \infty} F(x)=1$.
\item[2.]$F$ es monótona no decreciente.
\item[3.]$F$ es continua por la derecha.
\end{itemize}

\subsection{Teorema de la f. distribución.}
Una función $F:\mathbb{R} \to \mathbb{R}$ es la función de distribución de una v.a. X si y solo si verifica las tres propiedades. En base a lo anterior, también es válido:
\[F(a-) = P(X<a), \forall a \in \mathbb{R}\]
Además, se verifica lo siguiente:
\begin{itemize}
\item $\forall a,b \in \mathbb{R}, P(a < X \leq b)=F(b)-F(a)$.
\item $\forall a \in \mathbb{R}, P(X=a)=F(a)-F(a-)$.
\end{itemize}


\section{Variables aleatorias discretas. Función de masa.}
\subsection{Definición de v.a. discreta.}
Diremos que una variable aleatoria discreta es aquella que toma valores en un conjunto finito o numerable de puntos. Los puntos con probabilidad positiva se denominan \textbf{puntos de masa}.

\subsection{Definición de f. de masa.}
Sea $X: E \to \mathbb{R}$ una variable aleatoria que toma valores ${x_{1},...,x_{n},...}$ y sean $p_{i} = P(X=x_{i}), \forall i \in I = {1,2,...}$. El conjunto ${x_{i}, p_{i}}_{i \in I}$ tal que $\forall i \in I p_{i} \geq 0$ y $\sum_{i \in I} p_{i} = 1$ se llama \textbf{función de masa}.

\paragraph{}
Para este caso, la función de distribución de X se calcula como:
\[Para\: cada\: x \in \mathbb{R}, F(x)=P(X \leq x_{i}) = \sum_{x_{i} \leq x} P(X=x_{i}) = \sum_{x_{i} \leq x} p_{i}\]

La función de masa también caracteriza las v.a. discretas, por lo que usándola se puede calcular la probabilidad de cualquier suceso.

\subsection{Conclusión.}
En resumen:
\begin{itemize}
\item[1.]Tanto las funciónde masa como la función de distribución caracterizan las variables aleatorias discretas.
\item[2.]Conociendo una de ellas se puede obtener la otra con facilidad.
\item[3.]Ambas permiten calcular la probabilidad de cualquier suceso asociado a ese experimento aleatorio.
\end{itemize}


\section{Variables aleatorias continuas. Función de densidad.}
\subsection{Definición de v.a. continua.}
Diremos que una variable aleatoria $X:E\to\mathbb{R}$ es continua si existe una función $f$ integrable y no negativa tal que la función de distribución $F$ verifica:
\[F(x)=\int_{-\infty}^{x} f(t)dt,\: \forall x \in \mathbb{R}\]

\subsection{Definición de f. densidad.}
La función $f$ se llama función de densidad asociada a $x$, catacteriza a la v.a. y verifica las propiedades siguientes:
\begin{itemize}
\item[1.]$\int_{-\infty}^{\infty} f(x)dx = 1$.
\item[2.]$\forall a,b \in \mathbb{R}\: a<b,\: P(a < X \leq b) = \int_{a}^{b} f(x)dx$.
\item[3.]Si $X$ es una v.a. continua, $\forall a \in \mathbb{R}, P(X=a)=0$.
\item[4.]Toda función integrable $f$ que verifica que $f(x) \geq 0, \forall  \in \mathbb{R}$ y $\int_{-\infty}^{\infty} f(x)dx=1$ es la función de densidad de una v.a. cuya función de distribución es:
\[F(x)=\int_{-\infty}^{x} f(t)dt, \:\forall x \in \mathbb{R}\]
\item[5.]Si $X$ es una v.a. continua con función de distribución $F$, entonces $f(x)=F'(x)$ en los puntos en que F sea derivable y 0 en el resto.
\end{itemize}


\section{Esperanza matemática y varianza de una variable.}
\subsection{Esperanza matemática.}
Sea $X:E\to \mathbb{R}$ una variable aleatoria. Se llama \textbf{media, valor esperado} o \textbf{esperanza matemática} de $X$ y se escribe $E[X]$ a:
\begin{itemize}
\item Si $X$ es discreta con función de masa $\{x_{i}, P(X=x_{i})\}_{i \in I}$ y la serie converge:
\[E[X]=\sum_{i \in I} x_{i}P(X=x_{i})\]
\item Si $X$ es continua con función de densidad $f$ y existe la integral:
\[E[X]=\int_{-\infty}^{\infty}xf(x)dx\]
\end{itemize}

\subsubsection{Propiedades.}
\begin{itemize}
\item[1.]Si $X:E\to\mathbb{R}$ es una variable aleatoria y $g: \mathbb{R}\to\mathbb{R}$ una aplicación de manera que $Y=g(X)$ es también una variable aleatoria, entonces:
\begin{itemize}
\item Si $X$ es discreta con función de masa $\{x_{i},P(X=X_{i})\}_{i\in I}$ y la serie converge:
\[E[g(X)]=\sum_{i\in I}g(x_{i})P(X=x_{i})\]
\item Si $X$ es continua con función de densidad $f$ y existe la integral:
\[E[g(X)]=\int_{-\infty}^{\infty}g(X)f(X)dx\]
\end{itemize}
\item[2.]Si $g_{1}, g_{2}:\mathbb{R} \to \mathbb{R}$ son tales que existe $E[g_{1}(X)]$ y $E[g_{2}(X)]$, entonces para todos $a,b \in \mathbb{R}$:
\[E[ag_{1}(X)+bg_{2}(X)]=aE[g_{1}(X)]+bE[g_{2}(X)]\]
En particular:
\[E[aX+b]=aE[X]+b\]
\end{itemize}

\subsection{Varianza.}
Sea $X:E\to\mathbb{R}$ una variable aleatoria. Se llama \textbf{varianza} de $X$ y se escribe $V(X)$ a $V(X)=E \big[ (X-E[X])^{2} \big]$ si existe esta esperanza.\\
Para calcular la varianza, distinguimos según el tipo de variable aleatoria:
\begin{itemize}
\item Si $X$ es discreta con función de masa,
\[V(X)=\sum_{i\in I}(x_{i}-E[X])^{2}P(X=x_{i})\]
siempre que la serie sea convergente.
\item Si $X$ es continua con función de densidad,
\[V(X)=\int_{-\infty}^{\infty}(x-E[X])^{2}f(x)dx\]
si existe la integral.
\end{itemize}
